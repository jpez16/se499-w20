\section{Introduction}

\subsection{Purpose}
In terms of development, this project is considered abandoned. With that being said, the purpose of this report is to present to the reader the process \textbf{Team Cued} undergone during the winter 2020 term, as well as provide knowledge and structure for future groups to expand the project as a prospective SE490/491 project (SE2021, SE2022 etc.).
\subsection{What is a Habit?}
The Merriam Webster's dictionary definition of a habit is as follows: \textit{a settled tendency or usual manner of behavior} or as a more technical definition \textit{a behavior pattern acquired by frequent repetition or physiologic exposure that shows itself in regularity or increased facility of performance}. If we look at the second definition, we see the phrase "physiologic exposure that shows itself in regularity", remember this, it will come back later. 

\subsection{Project Scope}
The scope of the project is to create an application that allows users to discover and track habits that they a) want to form, or b) want to eliminate. Some examples of these types of habits include ”wanting to make my bed in the morning”, or ”stop biting my nails”. There are many habit managing apps that exist on the Google Play and Apple App stores allowing one to track similar things, however we found that they were either fairly lackluster or extremely bloated. On one end of the spectrum we discovered solutions that were glorified checklists, without much depth under the surface. On the other, we discovered various solutions that employed the concept of gamification, a well-known user experience concept that utilizes elements of game playing to encourage engagement with a product or service. The problem with these apps is that they encourage the user to perform (or not perform) the habit as a means to progress in the game, rather than for the lifestyle benefits. The common theme among all of these apps is that they allow the user to track habits, yet they lack  justification for why the app is allowing them to make lasting changes. The app lets the user check boxes or collect trinkets, thus giving them a sense of achievement in building their habit. In this work we specifically investigate the scientific proof behind these validations of positive re-enforcement.
