\section{Conclusion}
Developing this project turned out to be a challenge in multiple ways. With completely pivoting to this new project as late as January 2020, the amount of time to "symposium day" was extremely short. Fortunately, there was ample drive to complete the project, and within weeks of starting there were already tons of brainstorming occurring with the potential massive expansions we could add to the app. They were not implemented in the app in its current form, but to a future group wanting to take over the project they are definitely something to explore. On the technical side of things there was a strong bias towards Apple products and technologies due to a strong knowledge of them. For the purposes of the project they served very well. Ultimately however, the habit loop research performed by Charles Duhigh is what allowed this project to be so successful. From the research done by the team, it was clear that current habit app offerings are falling short. The goal of this project was to build an app not only allows people to track their habits using the habit loop, but encourage them to learn about the process and by doing so help them create long lasting sustainable habits. Ultimately, there is a amazing foundation for the design of an app that displays the habit research in an educational yet engaging format. By the end of the term it was fantastic that everything was able to come together and this document was produced in order to summarize the experience from not just a software engineering perspective, but also psychology research study as well.
\section{Next Steps} 
As the purpose of this paper was to summarize the experience and serve as somewhat of a repository of information for a future maintainer, if there is a group that is interested in taking this over especially in SE2021 and even SE2022 or beyond, please reach out to me at \url{jjpezzac@uwaterloo.ca}. Would love to take the opportunity to go through things with you. The first low fidelity prototype of the app is available on GitHub at: \url{https://github.com/cue-d/cued-app}. A basic demo that highlights some of the main features that have been developed as of now is available at: \url{https://github.com/justinpezzack/se499-w20/blob/master/basic-demo.mp4}.