\section{Conclusion}
Developing this project turned out to be a challenge in numerous ways. In completely pivoting to this new project as late as January 2020, the amount of time to ”symposium day” was extremely limited. Fortunately, there was ample drive to complete the project, and within weeks of starting there was already considerable amounts of brainstorming occurring, with a significant amount of potential further expansions to the app. They were not implemented in the current app presented in this work, but for prospective groups wanting to take over the project, they are certainly topics to explore. On the technical side of things, there was a strong bias towards Apple products and technologies due to a pre-existing knowledge of them. For the purposes of the project they served very well. Ultimately however, the habit loop research performed by Charles Duhigh is what allowed this project to propel towards success. From the research done by the team, it was clear that current habit app offerings fall short. The goal of this project was to build an app that not only allows people to track their habits using the habit loop, but also encourage them to learn about the process, and by doing so help them create long lasting sustainable habits. Nonetheless, there is an amazing foundation for the design of an app that displays the habit research in an educational, yet engaging manner. By the end of the term, it was fantastic that everything was able to come together, and this literature was produced in order to summarize the experience from not just a software engineering perspective, but a psychology perspective as well.
\section{Next Steps} 
The purpose of this paper was to summarize the project experience and document information for a future maintainer. If there is a group that is interested in taking this over this, especially in SE2021 and even SE2022 and beyond, please reach out to me at \url{jjpezzac@uwaterloo.ca}. I would love to take the opportunity to go through things with you. The first low fidelity prototype of the app is available on GitHub at: \url{https://github.com/cue-d/cued-app}. A basic demo that highlights some of the main features that have been developed as of now is available at: \url{https://github.com/justinpezzack/se499-w20/blob/master/basic-demo.mp4}.